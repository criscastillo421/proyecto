\section{Implementación}

Se implementaron los algorítmos BFE y LCMFLOCK basados en el pseudo-código publicado por \cite{vieira2009line}  y \cite{romero2011mining} respectivamente,
usando python version 3.

\subsection{BFE}

Este algoritmo se divide en dos partes: la primera parte, encontrar los discos que contienen el mayor número de puntos
dado un radio ($\epsilon$) y un número mínimo de puntos ($\mu$) dentro de ese radio para instante de tiempo. La segunda parte: encontrar
el número de puntos que se mueven juntos (flocks) en un tiempo mímimo ($\delta$).

Para la primera parte es necesarios la utilización tanto de diccionarios de datos como
estructuras kd-tree para la búsqueda del vecino más cercano, en esta implementación se uso la clase scipy.spatial.cKDTree de
SciPy \footnote{Scipy es un ecosistema basado en Python, software de código abierto para las matemáticas, la ciencia y la ingeniería.
\url{http://www.scipy.org/}} la cual proporciona  un índice dentro de un conjunto de puntos k-dimensionales que se pueden utilizar
para buscar rápidamente los vecinos más cercanos de cualquier punto.

\subsection{LCMFLOCK}

Este algoritmo usa la primera parte del algoritmo de BFE que es encontrar los discos maximos, en la segunda parte para abordar el problema
de combinatoria utiliza un enfonque de patron frecuente de mineria, en el cual se contruyo un diccionario de datos asociando la localizacion 
de los puntos de cada trayectoria con su respectivo disco en orden para generar una version transaccional del conjunto de datos. Este conjunto
es pasado como parametro junto con el umbral mínimo de soporte (min\_sup), para el algoritmo de LCM, disponible para descargar en \cite{FIMIHomep},
 están disponibles dos variantes del programa; LCM\_max y LCM\_closed los cuales recuperarán el conjunto máximo o cerrado de los patrones de 
 frecuencia, dependiendo del caso. La salida M es un archivo de texto sin formato, donde cada línea es un patrón máxima que 
 contiene un conjunto de ID's de discos separados por espacios.